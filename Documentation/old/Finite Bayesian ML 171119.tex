
\documentclass[12pt]{article}
% \documentclass{report}
% \documentclass[journal]{IEEEtran}

\usepackage{cite}
\usepackage{amsmath,amssymb,amsfonts}
\usepackage{algorithmic}
\usepackage{graphicx}
\usepackage{textcomp}
\usepackage{bm}

\usepackage[retainorgcmds]{IEEEtrantools}


\title{Bayesian Learning using a Non-Informative Prior over Finite-Dimensional Spaces}
\author{Paul Rademacher}
\date{October 20, 2017}



\DeclareMathOperator*{\argmin}{arg\,min}
\DeclareMathOperator*{\argmax}{arg\,max}


\begin{document}

\maketitle




\section{Introduction}

This report details a Bayesian perspective on stastical learning theory when both the input and outputs exist in finite dimensional spaces and a uniform prior distribution is used. To simplify the presentation, the first sections will assume that the predictions are not driven by any observable variable; subsequently, the results will be extended to the more practical case where we want to generate an estimate given observed data.

While the validity of Bayesian methods for statistical signal processing and machine learning has long been contended, the author believes it to be a justified approach that does not necessarily imply that the generative model is `random'; rather, it simply reflects the desire of the user to formulate risk as a weighted sum of learner performance across the space of models. 

The uniform, or `non-informative', prior is of specific interest because it reflects the user's lack of confidence that his/her data was generated using any specific distribution. Integrating a learner's risk with such a prior provides a Bayesian analogy to the ``No Free Lunch'' theorem; however, it will be shown that for a general loss metric, all learning functions \emph{do not} provide the same performance.

After examining the joint and conditional probability mass functions for the unobserved outputs and the training data, the results will be applied to two of the most common loss functions in machine learning: the squared error loss function (common for regression), and the 0-1 loss function (common for classification). Optimal learners will be presented and the loss as a function of space dimensions and volume of training data will be provided. Additionally, asymptotic results for these values will be discussed to provide insight into how well these learners perform for infinite input-output spaces and as the number of training examples increases. 




\section{Basic model: scalar estimate, no observations, identical decison space}


\subsection{Model}

In this simplified treatment, we want to design a learner to provide an estimate of unobserved variable $y \in \Omega = \{ \omega_1, \ldots, \omega_M \}$ using $N$ observed training examples $\mathcal{D} \in \Omega^N$. It is assumed that the output and training data are independently and identically distributed according to an unknown probability mass function (PMF), $\text{P}_{y|\bm{\theta}} \left(\omega_{m}\right) = \theta_m$, where,

\begin{equation}
\bm{\theta} = \left[ \begin{array}{c} \theta_1 \\ \vdots \\ \theta_M \end{array} \right]
\in \bm{\Theta} = \left\{ \bm{\theta} \in {\mathbb{R}^+}^M: \bm{1}^\text{T}\bm{\theta} = 1 \right\} \;. 
\end{equation}

FIGURES (simplex sets)

Our objective is to create a learning function $\hat{y}: \Omega^N \mapsto \Omega$ that minimizes a chosen risk function $\mathcal{R}(\hat{y}) \in \mathbb{R}^+$.  The algorithm designer makes two decisions regarding how risk is calculated. First, he/she chooses a loss function $\mathcal{L}: \Omega \times \Omega \mapsto \mathbb{R}^+$ that assigns a penalty dependent on both the unobserved output and the estimated value. Using this function, we calculate the conditional risk for a given model $\bm{\theta}$,

\begin{equation}
\mathcal{R}_{\bm{\theta}}(\hat{y}) = \text{E}_{\mathcal{D}|\bm{\theta}} \left[ \text{E}_{y|\bm{\theta}} \left[ \mathcal{L}(\hat{y}(\mathcal{D}),y) \right] \right] \;.
\end{equation}

Note the conditional independence between $y$ and $\mathcal{D}$. The second choice the designer has is how to formulate a scalar risk from the set of conditional risks $\mathcal{R}_{\bm{\theta}}(\hat{y}): \bm{\Theta} \mapsto \mathbb{R}^+$. An obvious choice is to integrate over $\bm{\Theta}$; if the designer has some prior knowledge, a weighting function $w(\bm{\theta}) \in \mathbb{R}^+$ can be used, such that the full risk function is,

\begin{equation}
\mathcal{R}(\hat{y}) = \int_{\bm{\Theta}} w(\bm{\theta}) \mathcal{R}_{\bm{\theta}}(\hat{y})\mathrm{d}\bm{\theta} \;.
\end{equation}

Since affine transformation of an objective function will not change its minimizer, we can freely scale the non-negative weighting function $w$ such that $\int_{\bm{\Theta}} w(\bm{\theta}) \mathrm{d}\bm{\theta} = 1$. As a result, $w$ is now a valid probability density function, allowing use of the Bayesian toolset and reducing the risk function to,   

\begin{equation}
\mathcal{R}(\hat{y}) = \text{E}_{\bm{\theta}}\left[  \mathcal{R}_{\bm{\theta}}(\hat{y}) \right] = \text{E}_{y,\mathcal{D}}\left[ \mathcal{L}(\hat{y}(\mathcal{D}),y) \right] \;.
\end{equation}

Now the unobserved variable $y$ and the observed training data $\mathcal{D}$ are treated as jointly distributed random variables. As mentioned previously, the two variables have the property of conditional independence given the model; as such, $\text{P}(y,\mathcal{D}) = \text{E}_{\bm{\theta}} \left[ \text{P}(y | \bm{\theta}) \text{P}(\mathcal{D} | \bm{\theta}) \right]$. This enables the following interpretation of the PMF of $y$ conditioned on the observed training data:

\begin{equation}
\text{P}(y|\mathcal{D}) = \text{E}_{\bm{\theta} | \mathcal{D}} \left[ \text{P}(y|\bm{\theta}) \right] \;,
\end{equation}

or,

\begin{equation}
\text{P}_{y|\mathcal{D}}(\omega_m) = \text{E}_{\bm{\theta} | \mathcal{D}} \left[ \bm{\theta}_m \right] \equiv \mu_{\bm{\theta}_m | \mathcal{D}} \;.
\end{equation}

At last, we express the optimal learning function,

\begin{equation}
y^* = \argmin_{\hat{y}} \mathcal{R}(\hat{y}) \;,
\end{equation}

and, more informatively, the specific estimate for a given training set, 

\begin{equation}
y^*(\mathcal{D}) = \argmin_{\omega \in \Omega} \text{E}_{y|\mathcal{D}}\left[ \mathcal{L}(\omega,y) \right] \;.
\end{equation}



\subsection{Properties of joint space $\text{P}(y,\mathcal{D})$}

Having formulated risk using a Bayesian treatment of the model $\bm{\theta}$, we now proceed to demonstrate the properties of the joint probability mass function (PMF), $\text{P}(y,\mathcal{D})$. Although the previous subsection provided general equations for any model distribution $\text{p}(\bm{\theta})$, the following discussion will assume that the model distribution is uniform over $\bm{\Theta}$. Note that the probability distribution function (PDF) for $\bm{\theta}$ is degenerate, since $\text{dim}(\bm{\Theta}) < M$ . Thus, we can choose to focus exclusively on the first $M-1$ elements of $\bm{\theta}$, which we will represent as,

\begin{equation}
\bar{\bm{\theta}} \in \bar{\bm{\Theta}} = \left\{ \bar{\bm{\theta}} \in {\mathbb{R}^+}^{M-1}: \bm{1}^\text{T}\bar{\bm{\theta}} \leq 1 \right\} \;.
\end{equation}

To express $\text{p}\left(\bar{\bm{\theta}}\right)$, we require a formulation of the hypervolume of set $\bar{\bm{\Theta}}$. This quantity is straightforward (PROOF???) to determine and allows us to provide the uniform distribution,

\begin{equation}
\text{p}\left(\bar{\bm{\theta}}\right)= (M-1)!,  \forall \bar{\bm{\theta}} \in \bar{\bm{\Theta}} \;.
\end{equation}

\begin{equation}
\text{p}(\bm{\theta}) = \text{p}(\bar{\bm{\theta}},\theta_M) = \text{p}\left( \theta_M | \bar{\bm{\theta}} \right) \text{p}\left(\bar{\bm{\theta}}\right)
= \delta\left( \theta_M,1-\bm{1}^\text{T}\bar{\bm{\theta}} \right) \cdot  (M-1)!
\end{equation}

First, we determine $\text{P}(\mathcal{D})$ -- extension to the joint distribution is straightfoward. Our starting point is expression of the PDF via the conditional independence of the training data for a given model $\bm{\theta}$,

\begin{equation}
\text{P}(\mathcal{D}) = \int_{\bm{\Theta}} \left[ \prod_{n=1}^N \text{P}(\mathcal{D}(n) | \bm{\theta}) \right] \text{p}(\bm{\theta}) \mathrm{d}\bm{\theta} \;.
\end{equation}

Observe that $\text{P}_{\mathcal{D}(n) | \bm{\theta}}(\omega_m) = \bm{\theta}_m$; as such, we can simplify the PMF to,

\begin{equation}
\text{P}(\mathcal{D}) = \int_{\bm{\Theta}} \left[ \prod_{m=1}^M \bm{\theta}_m^{\bar{\bm{N}}_m(\mathcal{D})} \right] \text{p}(\bm{\theta}) \mathrm{d}\bm{\theta} \;,
\end{equation}

where we have introduced the sufficient statistic $\bar{\bm{N}}: \Omega^N \mapsto \left\{ \bar{\bm{N}} \in \mathbb{N}^M: \bm{1}^\text{T}\bar{\bm{N}} = N \right\}$ FIXXX, which is formulated as,

\begin{equation}
\bar{\bm{N}}_m(\mathcal{D}) = \sum_{n=1}^N \delta(\mathcal{D}(n),\omega_m) \;.
\end{equation}

Note the dimensionality reduction from $N$ to $M$ provided from the sufficient statistic. Finally, as detailed in APPENDIX???, the training data PMF reduces to,

\begin{equation}
\text{P}(\mathcal{D}) = \binom{N+M-1}{M-1,\bar{\bm{N}}_1(\mathcal{D}),\ldots,\bar{\bm{N}}_M(\mathcal{D})}^{-1} 
= \binom{N+M-1}{M-1,N}^{-1} \binom{N}{\bar{\bm{N}}_1(\mathcal{D}),\ldots,\bar{\bm{N}}_M(\mathcal{D})}^{-1} \;.
\end{equation}

Note the use of multinomial and binomial coefficients, as well as the functional dependence on $\mathcal{D}$ only through $\bar{\bm{N}}$. It should be clear that since the PDF is inversely proportionate to the multinomial coefficient of $\bar{\bm{N}}$, training data sets are more probable when they are more "concentrated". 

This result can be used to illuminate the forms of all the PMF's of interest, namely $\text{P}(y)$, $\text{P}(y,\mathcal{D})$, and $\text{P}(y | \mathcal{D})$.  $\text{P}(y)$ follows immediately by evaluating $\text{P}(\mathcal{D})$ for $N=1$, leading to a uniform PMF,

\begin{equation}
\text{P}(y) = M^{-1} \qquad \forall y \in \Omega \;.
\end{equation}

DISCUSS MEAN OF THETA????????

The joint distribution between $y$ and $\mathcal{D}$ is expressed by simply extending $\text{P}(\mathcal{D})$ to $N+1$ data and treating $y$ separately in the multinomial coefficient:

\begin{equation}
\text{P}(y,\mathcal{D}) = \binom{N+M}{M-1,\bar{\bm{N}}_1(\mathcal{D})+\delta(y,\omega_1),\ldots,\bar{\bm{N}}_M(\mathcal{D})+\delta(y,\omega_M)}^{-1} \;.
\end{equation}

Combining these results, the PMF of the unobserved output conditioned on the training set is,

\begin{equation}
\text{P}_{y|\mathcal{D}}(\omega_m) = \frac{\text{P}(y,\mathcal{D})}{\text{P}(\mathcal{D})} = \frac{\bar{\bm{N}}_m(\mathcal{D})+1}{N+M} \;.
\end{equation}

A more interesting form for the conditonal PMF is,

\begin{equation}
\text{P}_{y|\mathcal{D}}(\omega_m) = \left(\frac{M}{N+M}\right) M^{-1} + \left(\frac{N}{N+M}\right) \frac{\bar{\bm{N}}_m(\mathcal{D})}{N} \;.
\end{equation}

Observe how the terms in parentheses act as weighting factors providing a convex combination of two distributions: a uniform distribution independent of the data (specifically, $\text{P}(y)$) and an empirical distribution based on the training data via $\bar{\bm{N}}$. The empirical distribution can be viewed as agnostic to any prior knowledge -- it is the maximum likelihood (ML) estimate of the model (see APPENDIX???),

\begin{equation}
\argmax_{\bm{\theta} \in \bm{\Theta}} \text{P}(\mathcal{D} | \bm{\theta}) = \frac{\bar{\bm{N}}(\mathcal{D})}{N} \;.
\end{equation}

The asymptotic behavior of this conditional PMF if of specific interest. As the $dim(\bm{\Theta}) = M$ increases relative to the number of training samples $N$, the PMF trends towards a uniform, "uninformative" distribution. Conversely, as the number of training points increases towards infinity, the PMF trends towards the empirical distribution. 

MODEL POSTERIOR?????




\section{Application to Common Loss Functions}

In this section, we will proceed to apply the previous results to determine specific optimal learners and assess performance for two of the most common loss functions in machine learning: the squared error function, common for regression problems, and the 0-1 function, common for classification problems. 

\subsection{Regession: the Squared-Error Loss}

The squared-error (SE) loss function is by far the most popular loss function for regression, or in fact any estimation problems where the variable of interest is continuous. This is due to its convexity, which allows easy determination of the minimizing function. We add an important generalization to this problem: we allow the estimate $\hat{y}$ DIFFERENT NOTATION???? to exist in a space that may differ from the space $\Omega$ of unobserved $y$. Such an approach is common in classification and general decision theory; here, we want to allow the variable to exist in a contiuous space to enable the closed-form solution for the minimizing solution.



%\appendix

%\section{Proof: Hypervolume of $\bar{\bm{\Theta}}$}

\end{document}


























