
\documentclass[11pt]{article} 
%\documentclass[journal]{IEEEtran}

\usepackage{cite}
\usepackage{amsmath,amssymb,amsfonts}
\usepackage{algorithmic}
\usepackage{graphicx}
\usepackage{textcomp}
\usepackage{bm}

\usepackage[retainorgcmds]{IEEEtrantools}




\title{MTI: CFAR vs Neural Networks}
\author{Paul Rademacher}


\begin{document}
\maketitle




\section{Introduction}

Traditional radar detection algorithms use CFAR to control the false alarm rate in the presence of interference that cannot be relaibly modelled. The theoretical basis for CFAR is that a signal is embedded in independent and identically distributed (i.i.d.) interference; however, im most practical MTI scenarios, such an assumption is unwarrented. As such, CFAR attempts to infer the power of the interference only from samples near the cell-under-test (CUT). While this allows greater flexibility in the application of CFAR, it incurs additonal detection error and fails to attempt any meaninful characterization of the interference.




\section{Model}

The model used is,

\begin{equation}
\bm{x}=A\bm{e}(f)+\bm{\omega} \in \mathbb{C}^N
\end{equation}

where [$\bm{e}(f)]_n = e^{j2\pi fn}$, signal amplitude $A\in \mathbb{C}$, and Doppler frequency $f \in [0,1)$. The interference $\omega$ is complex Gaussian with zero mean and covariance matrix $C_{\bm{\omega}} = F^{-1}DF$, where $F$ is the discrete Fourier transform (DFT) matrix and $D$ is a diagonal matrix whose $n^{th}$ diagonal entry is the user-defined power spectral density at frequency $f=(n-1)/N$.

As a result, we have the following conditonal probability distribution for the observed data:

\begin{equation}
p(\bm{x}|A,f) = \mathcal{CN}(\bm{x};A\bm{e}(f),C_{\bm{\omega}}).
\end{equation}

Radar detection can be easily cast into the form of binary hypothesis testing: $\mathcal{H}_0$ corresponds to interference only and $\mathcal{H}_1$ corresponds to a target in the presence of interference. The prior probabilities of these classes are considered to be equal ($P(\mathcal{H}_0)=P(\mathcal{H}_1)=1/2$); this ``uninformative" distribution reflects the uncertainty in the prior model. The Doppler frequency (normalized) can be effectively modelled by a uniform distribution $f \sim \mathcal{U}(0,1)$. The data probability function only depends on the class through the model for complex amplitude. Of course, if no target is present the amplitude should be zero: $p(A|\mathcal{H}_0) = \delta(A)$; this directly results in class conditional distribution $p(\bm{x}|\mathcal{H}_0) = \mathcal{CN}(\bm{x};\bm{0},C_{\bm{\omega}})$. Accurately modelling the signal amplitude when a target is present is more challenging. 


\subsection{Deterministic signal magnitude}

The complex amplitude is modelled as $A(\phi) = \alpha \text{exp}(j2\pi\phi)$, where $\alpha$ is the known magnitude and $\phi \sim \mathcal{U}(0,1)$.

\begin{equation}
p(\bm{x}|\mathcal{H}_1) = \left|\pi C_{\bm{\omega}} \right|^{-1} \text{exp}\left( -\bm{x}^\text{H} C_{\bm{\omega}}^{-1}\bm{x} \right) \int_0^1 I_\text{0}\left(2\alpha \left|\bm{e}^\text{H}(f) C_{\bm{\omega}}^{-1}\bm{x}\right| \right) \text{exp}\left( -\alpha^2\bm{e}^\text{H}(f) C_{\bm{\omega}}^{-1}\bm{e}(f) \right) df
\end{equation} 

\subsection{Random signal magnitude}

\begin{equation}
p(A|\mathcal{H}_1) =  \mathcal{CN}(A;0,\sigma_A^2)
\end{equation} 

\begin{equation}
p(\bm{x}|\mathcal{H}_1) = \int_0^1 \int p(\bm{x}|A,f) p(A|\mathcal{H}_1) dA df = \int_0^1 \mathcal{CN}(\bm{x};\bm{0},C_{\bm{\omega}}+\sigma_A^2 \bm{e}(f) \bm{e}^\text{H}(f)) df
\end{equation} 




\section{Likelihood Ratio Test}
The goal of the detector is to design a function $\hat{\mathcal{H}}: \mathbb{C}^N \mapsto \{\mathcal{H}_0,\mathcal{H}_1\}$ that minimizes the expected loss, or ``risk"; the loss function $\mathcal{L}: \mathcal{H} \times \mathcal{H} \mapsto \mathbb{R}^+$ is specified by the user.

\begin{equation}
\mathcal{R}(\hat{\mathcal{H}}) = \text{E}_{\bm{x},\mathcal{H}}\left[ \mathcal{L}\left( \hat{\mathcal{H}}(\bm{x}),\mathcal{H} \right) \right] \approx N^{-1}\sum_{i=1}^N \mathcal{L}\left( \hat{\mathcal{H}}(\bm{x}(i),\mathcal{H}(i) \right)
\end{equation}

For reference, we provide the Bayes optimal classifier, determined via the likelihood ratio test (LRT), to measure both CFAR and NN detectors against. Such an approach dictates both the optimum test statistic and the threshold to use. While the statistic is only dependent on the model assumptions provided above, the threshold is also dependent on the loss function. 

\begin{IEEEeqnarray}{rCl}
\mathcal{H}^*(\bm{x}) & = & \text{argmin}_{\hat{\mathcal{H}}} \text{E}_{\mathcal{H}|\bm{x}}\left[ \mathcal{L}\left( \hat{\mathcal{H}},\mathcal{H} \right) \right] \\ 
& = & \begin{cases} \mathcal{H}_0 & \text{if } \Lambda(\bm{x}) \leq T, \\ \mathcal{H}_1 & \text{if } \Lambda(\bm{x}) > T.\end{cases} 
\end{IEEEeqnarray}

\begin{equation}
\Lambda(\bm{x}) = \frac{p(\bm{x}|\mathcal{H}_1)}{p(\bm{x}|\mathcal{H}_0)} 
\end{equation}

\begin{equation}
T = \frac{(\mathcal{L}_{1,0}-\mathcal{L}_{0,0}) P(\mathcal{H}_0)}{(\mathcal{L}_{0,1}-\mathcal{L}_{1,1}) P(\mathcal{H}_1)}
\end{equation}


\begin{equation}
\Lambda(\bm{x}) = \frac{p(\bm{x}|\mathcal{H}_1)}{p(\bm{x}|\mathcal{H}_0)} \gtrless T = \frac{(\mathcal{L}_{1,0}-\mathcal{L}_{0,0}) P(\mathcal{H}_0)}{(\mathcal{L}_{0,1}-\mathcal{L}_{1,1}) P(\mathcal{H}_1)}
\end{equation}


\subsection{Deterministic signal magnitude}

\begin{IEEEeqnarray}{rCl}
\Lambda(\bm{x}) & = & \frac{p(\bm{x}|\mathcal{H}_1)}{p(\bm{x}|\mathcal{H}_0)} \\
& = & \int_0^1 I_\text{0}\left(2\alpha \left|\bm{e}^\text{H}(f) C_{\bm{\omega}}^{-1}\bm{x}\right| \right) \text{exp}\left( -\alpha^2\bm{e}^\text{H}(f) C_{\bm{\omega}}^{-1}\bm{e}(f) \right) df
\end{IEEEeqnarray}

\subsection{Random signal magnitude}

\begin{IEEEeqnarray}{rCl}
\Lambda(\bm{x}) & = & \frac{p(\bm{x}|\mathcal{H}_1)}{p(\bm{x}|\mathcal{H}_0)} \\
& = & \int_0^1 (1+\sigma_A^2 \bm{e}^\text{H}(f) C_{\bm{\omega}}^{-1} \bm{e}(f))^{-1} \text{exp}\left( (\sigma_A^{-2}+\bm{e}^\text{H}(f) C_{\bm{\omega}}^{-1} \bm{e}(f))^{-1}|\bm{e}^\text{H}(f) C_{\bm{\omega}}^{-1} \bm{x}|^2 \right) df
\end{IEEEeqnarray}




\section{CFAR}
The theoretical justification for CFAR follows from the assumption that the interference $C_{\bm{\omega}} = \sigma_\omega^2 I$, where $\sigma_\omega^2$ is unknown and unmodelled. Furthermore, $p(f)$ and $p(A|\mathcal{H}_1)$ are assumed unknown. As a result, the use of the generalized likelihood ratio test (GLRT) is necessary.

\begin{IEEEeqnarray}{rCl}
\Lambda(\bm{x}) & = & \frac{\max_{\sigma_\omega^2,A,f}p(\bm{x}|\mathcal{H}_1,\sigma_\omega^2,A,f)}{\max_{\sigma_\omega^2} p(\bm{x}|\mathcal{H}_0,\sigma_\omega^2)} \\
& = & \left( \frac{\min_f{\bm{x}^\text{H} \left(I - N^{-1}\bm{e}(f) \bm{e}^\text{H}(f)\right) \bm{x}}}{\bm{x}^\text{H}\bm{x}} \right)^{-N}
\end{IEEEeqnarray}

\begin{equation}
\Lambda^\prime(\bm{x})= \max_f \frac{N^{-1}\left|\bm{e}^\text{H}(f)\bm{x}\right|^2}{\bm{x}^\text{H}\bm{x} - N^{-1}\left|\bm{e}^\text{H}(f)\bm{x}\right|^2} \gtrless T^\prime
\end{equation}

Note that if we were testing for a specific Doppler frequency ($f$ known), the maximum operator would be excluded and $\Lambda^\prime(\bm{x}) \sim F(2,2(N-1))$ (the F distribution) under $\mathcal{H}_0$. This results since both the numerator and denominator are the norms of projections of Complex Normal random vectors (appendix???). This leads to the well known relationship between threshold $T^\prime$ and probability of false alarm $P_{fa}$:

\begin{equation}
T^\prime = P_{fa}^{-1/(N-1)} - 1
\end{equation}

To put this test statistic into a form more similar to that traditionally used in radar, we operate in the Fourier domain $\bm{y}=F\bm{x}$. The test becomes,

\begin{equation}
\Lambda^\prime(\bm{y})= \max_f \frac{\left|\left( N^{-1}F\bm{e}(f)\right)^\text{H}\bm{y}\right|^2}{\bm{y}^\text{H}\bm{y} -\left|\left( N^{-1}F\bm{e}(f)\right)^\text{H}\bm{y}\right|^2}
\end{equation}

If we further assume that $f=k/N$, where $k \in \{0,\dots,N-1\}$, we reduce to,

\begin{equation}
\Lambda^\prime(\bm{y})= \max_k \frac{\left|[\bm{y}]_k\right|^2}{\bm{y}^\text{H}\bm{y} -\left|[\bm{y}]_k\right|^2}
\end{equation}

Clearly, the denominator is just a scaled estimate of the interference power $\sigma_\omega^2$. Note that typical radar processing implements an independent test for each Doppler frequency sample $k$, allowing for multiple target detections. Also, observe that all frequency samples, except the sample being tested (the cell-under-test (CUT)), are used to estimate the interference power. Standard implementation of CFAR typically only uses a user-specified window of samples around the CUT; this reflects the understanding that the interference in the CUT will tend to be similar to "nearby" samples only. The samples used is represented with a set of indices relative to the CUT, such as $\mathcal{S} = \{-L,\dots,-G,G,\dots,L\}$, where $L$ controls the width of the window and $G$ defines the number of ``guard cells'' provided to exclude target energy from the interference power estimate.

\begin{equation}
\Lambda^\prime(\bm{y})= \max_k \frac{\left|[\bm{y}]_k\right|^2}{\sum_{i \in \mathcal{S}}\left|[\bm{y}]_{(k+i)(\text{mod}N)}\right|^2}
\end{equation}

This standard implementation of CFAR suffers from the assumption that the Doppler frequency is a multiple of $1/N$; when this is not satisfied, the signal is not 1-sparse in the DFT domain and signal is inadvertently included in the interference estimate. While guard cells mitigate this problem, it can be avoided using the approach below:

\begin{equation}
\Lambda^\prime(\bm{x})= \max_f \frac{\left|\bm{e}^\text{H}(f)\bm{x}\right|^2}{\sum_{i \in \mathcal{S}}\left|\bm{e}^\text{H}(f+i/N)\bm{x}\right|^2}
\end{equation}

As the maximum in the previous expression cannot be found analytically, we are forced to resort to a numerical approximation. For efficient computation, we first perform an oversampled DFT $\bar{\bm{y}}=\bar{F}\bm{x} \in \mathbb{C}^{MN}$, where $M$ is the oversampling factor. This enables a easily implemented CFAR algorithm that preserves accurate interference power estimates.


\begin{equation}
\Lambda^\prime(\bar{\bm{y}})= \max_k \frac{\left|[\bar{\bm{y}}]_k\right|^2}{\sum_{i \in \mathcal{S}}\left|[\bar{\bm{y}}]_{(k+Mi)(\text{mod}MN)}\right|^2}
\end{equation}

Note that these limited windows statistics have the distribution $F(2,2|\mathcal{S}|)$ under $\mathcal{H}_0$; as a result, the threshold (for fixed Doppler) is dependent on the cardinality of the index set, which is equal to or less than $N-1$. Since the actual LRT statistic used implements a maximum operator, the true target-absent GLRT does not follow the F-distribution. The PDF is approximated emprically and used to set a constant $P_{fa}$ threshold.




\section{Neural Network}




\end{document}





























